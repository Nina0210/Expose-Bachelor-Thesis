\documentclass[a4paper,12pt]{article}
\usepackage[utf8]{inputenc}
\usepackage[T1]{fontenc}       
\usepackage[polish,english]{babel} 
\usepackage{graphicx}
\usepackage{hyperref}
\usepackage{amsmath, amssymb}

\usepackage{geometry}
\geometry{a4paper, margin=1in}
\usepackage{natbib}
\usepackage{titlesec}
%\usepackage{multicol}

% Title Formatting
\titleformat{\section}{\large\bfseries}{\thesection}{1em}{}

\title{Memory-efficient PageRank on Apache Spark\\ via Random Walks}
\author{Nina Michalek\\TU Berlin, DOS}
\date{May 13, 2025}

\begin{document}

\maketitle

%\begin{multicols}{2}

\section{Introduction}
 Graphs are widely used to represent relationships between different entities for applications such as social networks and website links \cite{zhang_distributed_2021}. 
A wide range of graph algorithms have been developed to analyse these structures. \cite{malewicz_pregel_2010}\cite{low_distributed_2012}\cite{koch_empirical_2016}. Among the most influential is PageRank \cite{page_pagerank_1999}, an iterative algorithm introduced by Google to determine the importance of websites. Faced with the challenge of information retrieval posed by the rapid expansion of the World Wide Web in the late 1990s, PageRank provided a method of ranking web pages by measuring their relative importance based on the graph of the web. In this graph, the nodes are the web pages, while the edges represent links. An outedge is a link on a page and an inedge is a backlink. The core idea is that a page is considered important if it has inedges from important pages.\par
The PageRank value of a web page is calculated iteratively based on the values of all the pages that link to it. Linking pages pass on a fraction of their own value, proportional to their number of outbound links, to the target page. This calculation includes a damping factor that converges to a stable probability distribution, effectively representing the probability that a random user will end up on a particular page.
Today, PageRank is used not only in search engines, but in a wide range of applications such as social networks \cite{wu_efficient_2024}. These networks use PageRank to identify influential users, rank content in feeds and recommend people \cite{weng_twitterrank_2010}.\par
Despite its wide applicability, a fundamental challenge lies in the representation of graphs \cite{liu_fast_2015}. Typically, graphs are represented in matrices, which in the worst case require up to $O(n^2)$ of memory, where $n$ is the number of web pages \cite{wu_efficient_2024}. The transition matrix, which contains the probabilities of moving from one web page to another in a single step, presents a significant memory challenge as graph sizes grow.\par
To meet the demands of graph analytics, specialised graph processing frameworks have been introduced. These systems are designed to efficiently execute iterative algorithms, making graph analytics more practical. An example of such a system is GraphX \cite{xin_graphx_2013}, which is based on Apache Spark \cite{xin_graphx_2013}, a popular open source distributed dataflow system that enables large-scale data processing \cite{shanahan_large_2015}. GraphX acts as a graph processing framework that aims to bridge the gap between system-level optimisations in specialised graph engines and the flexible dataflow operations available in general-purpose systems \cite{jin_software_2022}. However, even with such frameworks, the memory consumption of algorithms such as PageRank can be enormous. Therefore, recent research has attempted to address the memory overhead in GraphX's PageRank algorithm due to the data representation, and proposes approximate computation to reduce memory consumption \cite{wu_efficient_2024}. 


 

%\section{Problem Description}
%The rapid expansion of the World Wide Web in the late 1990s posed a significant challenge to information retrieval as millions of pages made finding relevant and authoritative information increasingly difficult \cite{page_pagerank_1999}. With the launch of Google in 1998, PageRank was introduced as a core component of the search engine. PageRank is an algorithm that ranks web pages by measuring their relative importance based on the graph of the web. The nodes of the graph are the web pages, while the edges represent links. An outedge is a link on a page and an inedge is a backlink. A page is considered important if it has inedges from important pages. The value of a web page is calculated recursively based on the values of all the pages that link to it. Linking pages pass a fraction of their own value, proportional to their number of outbound links, to the target page. This is an iterative calculation with a damping factor that converges to a stable probability distribution.
%Today PageRank is used, not only in search engines, but in a wide range of applications such as social networks \cite{wu_efficient_2024}. The latter utilizes PageRank to identify influential users, rank content in feeds and recommend people \cite{weng_twitterrank_2010}. Despite its wide applicability, a fundamental challenge lies in the graph representation \cite{liu_fast_2015}. Typically, graphs are represented in matrices, which in the worst case require up to $O(n^2)$ of memory, where $n$ is the number of web pages \cite{wu_efficient_2024}. The transition matrix, containing the probabilities of moving from one web page to another in a single step, presents a significant memory challenge as graph sizes grow. 



\section{State of the Art}
Numerous approaches, such as Apache Spark's GraphX framework \cite{xin_graphx_2013}, have been introduced to address the memory overhead of the standard power iteration algorithm \cite{page_pagerank_1999}.
GraphX uses a vertex-centric model, strongly inspired by Pregel \cite{malewicz_pregel_2010}, which partitions the graph across a cluster to allow efficient distributed communication and state updates. It improves performance over general-purpose dataflow systems \cite{jin_software_2022}. However, GraphX's PageRank still has limitations on massive graphs, including significant memory requirements for storing the graph structure \cite{wu_efficient_2024}\cite{xin_graphx_2014}. Therefore, approximate PageRank algorithms have been proposed for practical large-scale analysis that trade off accuracy for efficiency \cite{wu_efficient_2024}. The following methods have been introduced:
\textbf{Iteration methods} \cite{xie_parameterized_2023-1}\cite{anikin_efficient_2022} accelerate the standard power iteration but still operate on the whole graph structure \cite{wu_efficient_2024}. 
\textbf{Sampling-based methods} estimate PageRank values by using smaller, sampled subgraphs \cite{bar-yossef_local_2008}\cite{chen_local_2004}. Despite reducing the size of the input data, these methods cannot estimate the PageRank values for the vertices outside the sampled set \cite{wu_efficient_2024}.
\textbf{Matrix approximation methods} use a low-rank approximated transition matrix to estimate PageRank values \cite{liu_fast_2015}\cite{benczur_feasibility_2005}. 
\textbf{Monte Carlo methods} estimate PageRank by simulating random walks on the graph. Instead of iteratively updating a full rank vector, these models simulate many random walks on the graph \cite{avrachenkov_monte_2007}. The value of a node is estimated by the number of walks that end at the vertex. In order to achieve high accuracy, these methods have to go through a large number of iterations \cite{wu_efficient_2024}. However, the memory required during the simulation can be controlled by the number of concurrent random walkers rather than by the total number of vertices \cite{avrachenkov_monte_2007}. Given the potential memory advantages by controlling the random walker count, Monte Carlo methods offer a promising direction for large-scale PageRank approximation.
 

\section{Approach}
The goal of this thesis is to implement a memory-efficient alternative to GraphX's iterative PageRank computation by using a Monte Carlo method in Apache Spark. The aforementioned memory advantage of Monte Carlo methods provides an approach that achieves lower peak memory consumption compared to GraphX's power iteration on the whole graph. Especially when processing large graphs. A trade-off between accuracy and performance due to memory constraints is accepted. 
The implementation uses RDDs (Resilient Distributed Datasets) to represent the graph structure and the dynamic states of concurrent random walkers. The simulation computes the next position for each walker over a given number of steps (k). In each step, the current walker state RDD is joined with the persisted graph structure RDD to determine available transitions. The PageRank random surfer model is then applied using standard Spark transformations. In this model each walker either follows a randomly selected outgoing link with probability d (damping factor) or teleports to a random node with probability 1-d. Dangling nodes are handled appropriately within the step. To manage memory effectively throughout the simulation, the RDD representing the previous walker state is explicitly unpersisted after the new state RDD is computed and persisted in each step. Finally, the PageRank scores are estimated based on the final walker states. The number of walkers and the step size have a direct impact on memory consumption and performance. The thesis evaluates the trade-offs by varying the number of walkers and the step size. The memory consumption, execution times and accuracy will be benchmarked against GraphX's PageRank implementation. \par
A variety of graph datasets will be used to develop and evaluate the methods. In addition to real-world data sets such as web graphs, social networks and citation networks, synthetic graphs enable a more systematically analysis of the methods. Different graph sizes will be considered to show the performance of the methods as graph size increases. Ideally, large graphs that strain the standard GraphX PageRank implementation should be included. 



\bibliographystyle{plain}  % Choose a citation format
\bibliography{references}   % Link to your BibTeX file

%\end{multicols}

\end{document}
